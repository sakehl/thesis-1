%
% sec-3-design.tex
%
% Embedded languages, stratified types, richly typed terms, representing
% programs. The Accelerate language.
%
% Accelerate CUDA backend. Code generation. Executing computations. Garbage
% collection. Caching.
%

\chapter{Design}
\epigraph{You see a wile, you thwart. Am I right?}%
{\textsc{---terry pratchett and neil gaiman}\\\textit{Good Omens}}

MMTC: Chapter 3 (Design): Maybe you should call the chapter ``Design and Related
Work''. It seems that you want to cover related work there and it is a good idea
to make that easy to find.

MMTC: More generally, there are two ways to do related work. Everything in one
related work chapter or have a related work section at the end of each chapter.
The latter works better if there is different sorts of related work for the
different chapters. This may apply here as, eg, the related work for embedded
languages and the related work for fusion has little overlap. (You can look at
my thesis for an example of that style.)


\begin{itemize}
\item other significant parallel/GPGPU programming models/languages; limitations
    to avoid/embrace and what features to steal (1.5 weeks)
    \begin{itemize}
        \item Repa \cite{Keller:2010er,Lippmeier:2011cd,Lippmeier:2012gx} (Haskell)
        \item SkeTo \cite{Matsuzaki:2011ew} (C++)
        \item list homomorphism based \cite{Sato:2009cq} (C++)
        \item Delite/LMS \cite{Rompf:2013er} (Scala)
        \item NDP2GPU \cite{Bergstrom:2012bi} (Haskell)
        \item Nikola \cite{Mainland:2010vj} (Haskell)
        \item Obsidian \cite{Svensson:2008a,Claessen:2012hl} (Haskell)
        \item Baracuda \cite{Larsen:2011fa} (Haskell)
        \item Jacket/ArrayFire \cite{AccelerEyes:vq} (Matlab)
        \item Anaconda Accelerate \cite{AnacondaAccelerate:2013vn} (Python)
        \item NOVA \cite{Collins:2013wn} (lisp?)
    \end{itemize}

    \item Why another parallel programming language? (1 day)

    \item Accelerate design (3 weeks)
        \begin{itemize}
            \item rank-polymorphic array language of collective operations
            \item programming model
            \item algorithmic skeletons
            \item stratified language: using the type system to exclude nesting
            \item richly typed terms; environments \& types (type safe evaluator)
            \item types help catch bugs in the compiler; important since
                compilation happens at program \emph{runtime}
            \item surface vs.\ internal (core) languages
            %\item representing different constructs (c.f. nesting?)
            \item examples: operator expressiveness (application driven)
        \end{itemize}

\end{itemize}


