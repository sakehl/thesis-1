%
% Background / Literature Review
%
% The literature review is an organisational pattern that combines both summary
% and synthesis. It may give a new interpretation of old material, combine old
% and new interpretations, or trace the intellectual progression of a field. It
% should be organised around ideas, not expand each source individually.
%
% - What themes and issues connect the sources together?
% - What solutions are present, which aspects are missing?
% - What is the trend in the field?
%
% http://blog.accelereyes.com/blog/2013/03/07/cpu-processing-trends-for-dummies/
%

\chapter{Background}
\epigraph{Why waste time learning, when ignorance is instantaneous?}
{\textsc{---calvin and hobbes}}
\label{ch:background}

%
% Progression:
%  1. Technology trends, move towards increasing parallelism
%  2. Parallel computing, introduce NDP and speedup/code -> inf promise
%  3. GPGPU. Short history. Problems, need for reuse, abstraction from HW
%  4. Develop a need/motivation to combine (2) and (3)
%
% Organisation:
%  1. Introduction: quick idea of the topic, central theme
%  2. Body: sources thematically
%  3. Conclusion and recommendations
%
%  + Models and Languages for Parallel Computing
%    + Challenges to Parallel Computing
%    + Approaches to Parallel Computing
%    + Evaluating Parallel Programs
%
%  + Parallel Programming for Graphics Processing Units
%  + CUDA
%    + Hardware Model
%    + Programming Model
%    + Execution Control
%    + Memory Model
%
%  + Motivation
%

\begin{itemize}
    \item Functional programming 101
        \begin{itemize}
            \item Haskell 101?
            \item bulk programming; function composition
            \item denotional semantics?
        \end{itemize}

    \item Parallel programming 101 (steal from literature review)
        \begin{itemize}
            \item parallel vs. concurrent
            \item data parallelism
                \begin{itemize}
                    \item example parallel programming operators (fold, scan)
                    \item parallel interpretation of ostensibly serial
                        operations
                \end{itemize}
            \item evaluating parallel programs
                \begin{itemize}
                    \item work and depth complexity
                    \item parallel speedup
                    \item wall clock time!
                \end{itemize}
        \end{itemize}

    \item CUDA programming 101
        \begin{itemize}
            \item GPGPU programming
                \begin{itemize}
                    \item why bother?
                    \item performance vs. development time
                \end{itemize}
            \item programming model
                \begin{itemize}
                    \item SIMD architecture
                    \item kernels, thread blocks
                    \item memory \& coalescing
                \end{itemize}
        \end{itemize}

    \item Embedded languages
        \begin{itemize}
            \item advantages: using tools/features of host language
            \item deep vs. shallow. \url{http://skillsmatter.com/podcast/scala/making-edsls-fly}
            \item DSLs in Haskell
                \begin{itemize}
                    \item simple (untyped) DSL
                    \item GADTs to encode type safety / invariants
                    \item type-safe evaluator
                    \item itemize design requirements / goals of this language
                \end{itemize}
        \end{itemize}

\end{itemize}



Modern microprocessor technology is constructed from millions of interconnected
switching devices, called transistors. As process technology advances, these
transistors, and the interconnections between them, can be fabricated in a
smaller area. The challenge for microprocessor architects is to translate this
increase in resources into an increase in performance.

Unfortunately, computers are no longer getting faster. Instead, we are being
offered computers containing more and more CPUs, each of which is individually
not substantially faster than those of the previous generation. As the number of
processors increases, it will become increasingly important for programmers to
utilise these additional processing elements.

Processor designs are clearly evolving in one direction: towards massive
parallelism and many-core architectures. Unfortunately, the model of programming
used in mainstream languages was developed many years ago, when programs were
sequential and architectures were relatively simple. As hardware architectures
become larger and more complex, these traditional low-level approaches to
parallel programming become less and less tractable for real applications.

This section \ldots


\section{Parallel Programming}

Programming parallel computers is an extremely challenging task. This is true
even for expert computer programmers, let alone for scientists in other
disciplines whose computations are often what drive the acquisition of
high-performance machines. For many years, parallel computing has waned in and
out of favour, alternatively seen as the solution to all computational
limitations or a waste of time, given the rate at which processor and memory
speeds improve. These changing perceptions are the result of many factors,
including the programming environments available to users, the types of
supercomputing hardware on offer from vendors, and continual changes in the
focus of the academic community as to what are the ``hot'' problems of the time.

Nevertheless, the promise of parallel computing remains the same today as it was
at its inception. If users can buy sequential computers with gigabytes of
memory, imagine how much faster their programs could run if \emph{p} of these
machines were working cooperatively; imagine how much larger a problem could be
solved if the memories of these \emph{p} machines were used collectively.

The challenges to realising this potential faces two main difficulties; the
hardware, and the software. The former asks ``how do I build a parallel computer
that will allow these \emph{p} processors and memories to cooperate
efficiently?'' Then, given such a platform the latter asks ``how do I express my
computation such that it will utilise these \emph{p} processors and memories
effectively?''

In recent years there has been a growing awareness that while the parallel
computing community can build machines that are reasonably efficient and
affordable, the population of users that can effectively program these machines
comprises only a small fraction of those who can program traditional sequential
computers. Moreover, even the best parallel programmers can not do so without
significant effort. Parallel programming is still seen by many as an arcane
skill, a black art; best avoided if possible.

However, recent changes to the landscape of computing hardware will likely force
even mainstream application developers to abandon the ways of sequential
programming. In order to continually increase performance, compiler and
processor designers have struggled to to automatically exploit the implicit
parallelism present in serial programs. Using this technique, instructions can
be rescheduled in order to best exploit the multiple functional units available
in superscalar architectures. However, experience has shown that most serial
programs contain limited implicit parallelism \cite{Wall:1991}. While
techniques exist that can raise this limit somewhat, they come at the cost of
increased power to performance ratios and hardware complexity. Automatic
extraction of parallelism without explicit assistance from the programmer has
reached the point of diminishing returns. Including multiple cores on a single
chip has now become the dominant mechanism for scaling processor performance,
and there is renewed interest in parallel programming models, languages and
platforms.

At the same time, one must concede that programming a parallel machine is
inherently more difficult than sequential programming, as programmers must think
not only of correctness but concurrency and communication as well. It
undoubtedly complicates a programmer's life, but there is no silver bullet.
However, a good programming model should be able to unburden the programmer from
dealing with trivia, automating the more tedious aspects of mapping the
computation to the target hardware. Ideally, the programmer should focus on
%making major policy decisions and on
designing a correct and efficient algorithm.


\section{Haskell}
%
% Possible; Haskell compared to:
%   + static languages
%   + dynamic languages
%

\indexe{Haskell} is a general-purpose \emph{purely functional programming
language} with \emph{non-strict semantics} \index{non-strict semantics|see{lazy}}
and \emph{strong static typing}\index{typing!static}. In an imperative programming
language you get things done by giving the computer a sequence of actions to
perform and letting it execute them. While executing these statements the
computer is typically allowed to change some state: setting some variable
\texttt{ans} to $42$, doing some stuff, and then changing \texttt{ans} to
something else. In contrast, purely functional programming is not so much
concerned with telling the computer what to \emph{do} in so much as telling it
what stuff \emph{is}. The factorial of a number is the product of all the
numbers from one to that number; the sum of a list of numbers is the first
number plus the sum of the rest of the numbers; and so on. This is expressed in
the form of \emph{functions}, and programs can then be thought of as a sequence
of functions applying a series of transformations on data.

Haskell is a \indexe{pure} language; if you say some variable \texttt{ans} has
the value $42$, you can not later say that it has a different value. We assert
that pure functions like this do not have any \indexe{side effects}: the only
thing they can do is to perform some calculation on their inputs and return a
result. This might at first seem limiting, but has a rather nice consequence: if
a function is called twice with the same inputs, it will always return the same
result. This is known as \indexe{referential transparency} and not only allows
the compiler to reason about the behaviour of the program, but also allows the
user to deduce the \indexe{correctness} of a function and to build complex
functions by gluing together simpler functions.

Haskell is a \emph{statically typed}\index{typing!static} language; when you
compile your program the compiler knows which piece of code is a number, which
is a string, and so on. As such many possible errors are caught at compilation
time; the compiler will complain if you attempt to add a number to a string, for
example. Haskell uses a very good system of type inference based on
Hindley--Milner \textcolor{blue}{[citation needed]}, so that you don't have to
explicitly label each piece of code with a type.

Haskell is a \indexe{lazy} language; unless told otherwise it will not
explicitly evaluate functions or calculate values until a result is actually
required. Lazy evaluation might seem to have some spooky effects, but goes in
hand with the idea of referential transparency and profoundly affects how one
writes programs.

While Haskell encourages the pure, lazy style of functional program, this is
merely the default option. Haskell supports stateful functions, using distinct
types for functions with side effects orthogonal to the type of pure functions,
ensuring the two can not be conflated. While the focus of the language is
writing statically typed programs it is possible, although rare, to write
dynamically typed programs as well.


% \subsection{Syntax}
% basic crash course necessary?

% \subsection{Style}


\section{CUDA}

