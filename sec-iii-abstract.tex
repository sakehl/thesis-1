%          File: sec-iii-abstract.tex
%       Created: Wed 07 Jun 2006 11:08:51 PM EST
% Last Modified: Sat 28 Oct 2006 04:59:03 PM EST

\chapter{Abstract}
\markboth{Abstract}{}

Computers are no longer getting faster. Instead, including multiple cores on a
single chip has become the dominant mechanism for scaling processor performance,
and the exponential growth in the number of cores on a single processor is
expected to lead in a short time to mainstream computers containing hundreds of
devices. Unfortunately, programming parallel computers is known to be an
extremely challenging task, even for expert computer programmers.

In order to achieve improved single-application performance on such processors,
an appropriate programming model is needed which can expose parallelism and
address scalability. Commodity graphics processing units, which have evolved
into flexible general-purpose parallel architectures, can provide some insight
into many of the challengers developers will face; finding parallelism within
our applications to satiate available hardware, and rationalising the
interactions of large numbers of concurrent threads.

This thesis presents the design, implementation, and evaluation of a new
domain--specific language embedded within Haskell for flat data-parallel array
computations executed on graphics processing units. The design concentrates on
providing a high--level abstraction that composes computations that run
efficiently according to the strengths and weaknesses of the underlying
hardware, without forcing the programmer to recast their algorithm to fit within
the architecture.

The final design \ldots\ and benchmarking shows that the goals of the design were
\ldots

