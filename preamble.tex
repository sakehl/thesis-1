
\documentclass[11pt,a4paper,twoside]{book}

% Imports
% ------------------------------------------------------------------------------

\usepackage[T1]{fontenc}
\usepackage{ifpdf}
\ifpdf
\RequirePackage{hyperref}
\hypersetup{
    pdftitle={\doctitle},
    pdfsubject={\docsubtitle},
    pdfauthor={Trevor L. McDonell},
    colorlinks=false
}
\pdfcompresslevel=9
\fi

\usepackage{a4wide}
\usepackage{amsmath}
\usepackage{amssymb}
\usepackage{calc}
\usepackage{caption}
\usepackage[usenames,dvipsnames]{xcolor}
\usepackage{doi}
\usepackage{epigraph}
\usepackage{fancyhdr}
% \usepackage{tipa}
\usepackage{fontspec}
\usepackage{graphicx}
\usepackage[all]{hypcap}
\usepackage{ifthen}
% \usepackage{dejavu}
% \usepackage{inconsolata}
% \usepackage{sourcecodepro}
% \usepackage{lastpage}
\usepackage{listings}
\usepackage{longtable}
\usepackage{makeidx}
\usepackage{marginnote}
\usepackage{mathrsfs}
\usepackage{mflogo}
\usepackage[numbers,sort&compress,nonamebreak]{natbib}
% \usepackage{pdfsync}
\usepackage{rotating}
\usepackage{setspace}
% \usepackage{showidx}
\usepackage[color,draft]{showkeys}
\usepackage{subfig}
\usepackage{titlesec}
\usepackage[width={autoauto}]{thumbs}
\usepackage{upquote}
\usepackage{xfrac}
\usepackage{textcomp}

\onehalfspacing
%\doublespacing


% Use symbols instead of numbers for footnote marks
% ------------------------------------------------------------------------------

%\renewcommand{\thefootnote}{$\fnsymbol{footnote}$}


% Bibliography
% ------------------------------------------------------------------------------

%\bibliographystyle{apsrev}             % citation order style
%\bibliographystyle{apsrmp}             % author/year style
\bibliographystyle{abbrvnat}
\providecommand{\bibfont}{\small}


% Figures and captions
% ------------------------------------------------------------------------------

\captionsetup{margin=\parindent/2,font=small,labelfont=bf}
% \captionsetup[subfloat]{margin=0pt,labelfont=default}

% Reduced bias to floats on their own page
%
% \renewcommand{\topfraction}{0.85}
% \renewcommand{\textfraction}{0.1}
% \renewcommand{\floatpagefraction}{0.75}


% Fancy header style
% ------------------------------------------------------------------------------

\newcommand{\sectionname}{Section}
\pagestyle{fancy}
\fancyhf{}
\renewcommand{\headrulewidth}{0pt}
\renewcommand{\chaptermark}[1]{\thispagestyle{empty}\markboth{#1}{}}
\renewcommand{\sectionmark}[1]{\markright{#1}}
\fancypagestyle{plain}{\fancyhf{}}
\fancyhead[EL]{\small\textsc{\thepage\hspace{2\medskipamount} \leftmark}}
%\fancyhead[ER]{\small\textsc{\chaptername\ \thechapter}}
%\fancyhead[OL]{\small\textsc{\sectionname\ \thesection}}
\fancyhead[OR]{\small\textsc{\rightmark\hspace{2\medskipamount} \thepage}}

\newcommand{\doccolor}{Bittersweet}
\newcommand{\docminorcolor}{Gray}


% Extra depth in numbered sections
% ------------------------------------------------------------------------------

\setcounter{secnumdepth}{3}
%\setcounter{tocdepth}{3}
%\setcounter{lofdepth}{2}


% Index
% ------------------------------------------------------------------------------

\makeindex
\providecommand{\indexe}[1]{\index{#1}\emph{#1}}
\providecommand{\indext}[1]{\index{#1}#1}


% Citations
% ------------------------------------------------------------------------------

\let\Cite\cite
\renewcommand{\cite}{\nolinebreak\Cite}


% Keys and margin notes
% ------------------------------------------------------------------------------

% Showkeys
% texmf-dist/tex/latex/graphics/dvipsnam.def
%
% \definecolor{refkey}{cmyk}{0.0,0.89,0.94,0.28}        % BrickRed
% \definecolor{labelkey}{cmyk}{0.0,0.89,0.94,0.28}

% notes, margin notes, etc. delete for final mode
%
%\let\oldmarginnote\marginnote
%\providecommand{\marginnote}{\oldmarginnote }
\renewcommand*{\marginfont}{\color{BrickRed}\ttfamily\scriptsize}

\newcommand{\derp}{\textcolor{BrickRed}{derp}}
\newcommand{\note}[1]{\textcolor{BrickRed}{#1}}


% Code listings
% ------------------------------------------------------------------------------

% Fix spacing between chapter groups in the list of listings
%
% Unfortunately this _always_ adds the space, so the first entry is lower than
% it should be. We hack around this by adding some negative space to begin with.
%
\let\Chapter\chapter
\def\chapter{\addtocontents{lol}{\protect\addvspace{10pt}}\Chapter}
\addtocontents{lol}{\protect\vspace*{-20pt}}

\newfontfamily\sourcecodepro{Source Code Pro}


% Define the language styles we will use
%
\lstset{
    frame=single,
    rulecolor={\color[gray]{0.7}},
    numbers=left,
    basicstyle=\sourcecodepro,
    numberstyle=\color{Gray}\tiny\it,
    commentstyle=\color{MidnightBlue}\it,
    stringstyle=\color{Maroon},
    keywordstyle=[1],
    keywordstyle=[2]\color{ForestGreen}\bf,
    keywordstyle=[3]\color{Bittersweet},
    keywordstyle=[4]\color{RoyalPurple},
    captionpos=b,
    aboveskip=2\medskipamount,
    xleftmargin=0.5\parindent,
    xrightmargin=0.5\parindent,
    flexiblecolumns=false,
%   basewidth={0.5em,0.45em},           % default {0.6,0.45}
    escapechar={\%},
    texcl=true                          % tex comment lines
}

\lstloadlanguages{Haskell}

\lstdefinestyle{haskell}{
    language=Haskell,
    upquote=true,
    basicstyle=\scriptsize\sourcecodepro,
    deletekeywords={case,class,data,default,deriving,in,instance,let,of,type,where},
    morekeywords={[2]class,data,default,deriving,family,instance,type,where},
    morekeywords={[3]in,let,case,of},
    literate=
        {\\}{{$\lambda$}}1
        {\\\\}{{\char`\\\char`\\}}1
        {->}{{$\rightarrow$}}2
        {>=}{{$\geq$}}2
        {<-}{{$\leftarrow$}}2
        {<=}{{$\leq$}}2
        {=>}{{$\Rightarrow$}}2
        {|}{{$\mid$}}1
        {forall}{{$\forall$}}1
        {exists}{{$\exists$}}1
        {...}{{$\cdots$}}3
%       {`}{{\`{}}}1
%       {\ .}{{$\circ$}}2
%       {\ .\ }{{$\circ$}}2
}

\lstdefinestyle{cuda}{
    language=CUDA,
    basicstyle=\scriptsize\sourcecodepro,
    keywordstyle=[1]\color{ForestGreen},
    keywordstyle=[2]\color{Bittersweet},
    keywordstyle=[3]\color{BrickRed},
    keywordstyle=[4]\color{RoyalPurple},
    keywordstyle=[5]\color{Cyan}
}

\lstdefinestyle{ptx}{
    language=CUDA,
    basicstyle=\scriptsize\sourcecodepro,
    mathescape=false,
    escapeinside={(@*}{*@)}
}

\lstdefinestyle{inline}{
    style=haskell,
    frame=none,
    numbers=none,
    basicstyle=\small\sourcecodepro
}


% Shorthand for inline code fragments
%
\newcommand{\code}[1]{\lstinline[style=inline]{#1}}


% Chapter page style
% ------------------------------------------------------------------------------

% Epigraphs
%
\renewcommand{\textflush}{flushright}
\setlength{\epigraphwidth}{0.85\textwidth}
\setlength{\epigraphrule}{0pt}
\setlength{\afterepigraphskip}{1.5\baselineskip}


% Titlesec
%
\titleformat{\chapter}[display]%
    {\bfseries\Large}%
    {\vspace{-4ex}\filleft\Huge\thechapter}%
    {4pt}%
    {\filleft\titlerule[0.8pt]\vspace{1pt}\titlerule\vspace{2ex}}%
    [\vspace{2ex}{\titlerule[0.8pt]}\vspace{1pt}\titlerule]

