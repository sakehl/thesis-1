%
% sec-1-introduction.tex
%

\chapter{Introduction}
\epigraph{Come then, and let us pass a leisure hour in 
storytelling,\\and our story shall be the education of our heroes.}%
{\textsc{---plato}\\\textit{Republic}, \textsc{book ii}}


Over the last ten years, an interesting trend in computing has emerged. General
purpose CPUs, such as those provided by Intel, IBM and AMD, have increased
performance substantially, but with nowhere near the increases seen in the late
1980's and early 1990's. To a large extent, single threaded performance
increases have tapered off due to the low level of inter-process communication
in general purpose workloads, and the physical limitations of power dissipation
for integrated circuits. The additional millions and billions of transistors
afforded by Moore's Rule\footnote{Moore's actual prediction in 1965 referred
only to the number of devices that could be fabricated on a single die, and that
this quantity would increase by fifty percent each year.} are simply not very
productive in increasing the performance of single--threaded code.

At the same time, the commodity \indexe{graphics processing unit} (GPU) has been
able to use this ever increasing transistor budget effectively, geometrically
increasing rendering performance, since rendering is an inherently parallel
operation. Historically fixed function pipelines, modern graphics architectures
have evolved to become increasingly flexible as well as powerful, consisting of
an expressing set of general purpose computation resources, with some fixed
function units on the side.

Parallel programming models fall into two broad categories. Task parallelism
emphasises a distributed model for independent processes, whereas data
parallelism specifies a single computation which is applied simultaneously
across a large number of data elements. While task parallelism typically does not
scale well with the size of the problem, data parallelism provides one of the
most promising approaches to making efficient use of parallel hardware.

Despite their attractiveness for computationally intensive operations, accessing
this performance has remained elusive for all but a few applications. The
arithmetic power of the GPU is a result of a highly specialised and constrained
architecture, resulting in programs which rely heavily on the CPU to handle
difficult parts of control and data flow. However, the CPU--GPU link is
relatively slow, engendering high communication latencies which radically impair
performance. To be practicable, the GPU programming model must efficiently
support messy boundary conditions, irregular data and control flow patterns, and
provide an elegant mapping from the application to the underling architecture,
without forcing the programmer to recast their algorithm. Support for
\indexe{nested data parallelism}\cite{Blelloch:1994,Blelloch:1996} (NDP) would
provide a first step towards a viable GPU programming model, unshackling vast
computational resources from the small application niche for which they were
originally designed.

